\section{Jakub Zięba}
 
\begin{figure}[htbp]
    \centering
    \includegraphics[width=0.8\textwidth]{pictures/image_jakub_zięba.png}  
    \caption{Frame from Twin Peaks}
    \label{fig:twin_peaks}
\end{figure}

I've added a picture of \textbf{Audrey Horne} from Twin Peaks, who said a very famous line in \text S1.Ep4: Rest in Pain (see Figure~\ref{fig:twin_peaks}). I think we can relate to them.Therefore, now I will try to prove whether a given algebraic structure is an abelian group:

Prove that the group $(A,\cdot)$, where $\cdot$ is a simple multiplication in the set of complex numbers. We define the set A as follows: $A =$ $\{{z \in \mathbb{C}:z=\sqrt[4]{1}\}}$ 

Computing the roots of a complex number $z$, we obtain complex numbers that belong to the set $A =$ $\{{1,-1,i,-i\}}$. The formula for the k-th root of any complex number $a$ is defined as follows:
\[{\omega_k}=\sqrt[n]{|a|}(\cos\frac{\phi+2k\pi}{n}+\sin\frac{\phi+2k\pi}{n}), k\in\mathbb{Z}\]

A set $A$ with binary operation $\cdot$ defined on it is called an abelian group if it satisfies the following axioms:

\begin{enumerate}
\item \underline{Internality}: $\forall x,y \in A $ : $x \cdot y \in A$

\input{tables/tab_Jakub_Zięba}
\item \underline{Associativity}: $\forall x,y,w \in A $ : $(x \cdot y)\cdot w = x\cdot (y\cdot w)$

The $\cdot$ operation is associative in any set.
\item \underline{Identity element} $e$: $\exists e\in A \;\; \forall x \in A $ :
\begin{itemize}
\item $x \cdot e = x $
\item $e \cdot x = x $
\end{itemize}
The identity element of the $\cdot$ operation in A is the number $e=1$

\item \underline{Inverse element} $x^{-1}$: $\forall x\in A \;\; \exists x^{-1} \in A $ :
\begin{itemize}
\item $x \cdot x^{-1} = e $
\item $x^{-1} \cdot x = e $
\end{itemize}
The inverse element of the $x$ in $\cdot$ operation in A is defined $x^{-1}=\frac{e}{x}$

\item \underline{Commutativity}: $\forall x,y \in A $ : $x \cdot y = y \cdot x$

The $\cdot$ operation is commutative in any set.
\end{enumerate}

\newpage